\documentclass{article}

\usepackage{paracol}
\usepackage{fontspec}
\usepackage{setspace}
\usepackage{csquotes}%textelp
\usepackage{titlesec}
\usepackage{comment}
\usepackage{paracol}
\usepackage[a4paper, total={4in, 8in}]{geometry}

\usepackage{dramatist}

\titlespacing*{\section}
{0pt}{5.5ex plus 1ex minus .2ex}{4.3ex plus .2ex}
\titlespacing*{\subsection}
{0pt}{5.5ex plus 1ex minus .2ex}{4.3ex plus .2ex}

\setlength{\parindent}{0.1cm}
\setmainfont[Ligatures=TeX]{ebgaramond}

\begin{document}

\setstretch{1.25}
\renewcommand{\baselinestretch}{1.25}

\section{protagonistes}

\textit{Nous racontons ici l'histoire de notre protagoniste Cédric qui oscille
  entre romantisme et son anti-thèse qu'il ne peut encore formuler. Il y a des
  poèmes, des faits des histoires mais surtout une contradiction entre le faire
  et l'expérience. Comment trouver sa voix entre la machine rationalisée et le
  rêve schizoïde. Nous commençons à un point assez central, sa dernière année de
  bac en Math. Il partage un appartement rue Christophe-Colomb avec deux
  musiciens que l'on pourrait aussi qualifier d'énergumenes. Il dispose d'un
  cercle d'amis fidèles et variés, entouré d'un cercle concentrique de
  connaissances énergisantes et divertissantes.}\\


\textit{Jean qui est ingénieur et fait le tour du monde, il sort d’où on sait pu
  trop, la Zambie, toujours la Zambie et la Malaysie surtout d’où il revient
  avec ses histoires abracadabrantes, une légère barbe hirsute, de nouvelles
  normes culturelles et une nouvelle personnalité qui vient se graffer sur ce
  qu’était Jean pré-nouveau voyage qui change toujours mais toujours grand et
  blond et blanc,}\\

\textit{Joe et ses lunettes rondes et son humour décapant, son charisme de dents
  tachées [...] ses cheveux gras et lisse, ses yeux sombres et son teint olive,
  ses larges poignets ses yeux olive et son regard ombrageux, son je-men-
  foutisme maintenant garni d’un concluant salaire à la radio de Radio-Canada,}\\


\textit{Jolie}\\

\textit{Jade}\\

\textit{Galiffee}\\

\textit{Dave}\\

\clearpage

 \section*{Tache}
 nuages clairs une précision obscure des arbres valsent un chien court un
 chien aboi le chat l’attaque Jolie roule Jolie chante; un vélo vole des portes
 claquent des cloches sonnent des casiers qui grincent mordillent les doigts
 des lumières blanches qui tachent l’endroit ; c’est long c’est long ça finit
 pu de traverser les halles du  CEGEP de Rimouski Jolie y passe de
 longues journées à écouter des quadratiques des Habsbourg des martyrs brûlant
 dans les feux de joie d’Iroquois des électrons de valence des normes éthiques
 des philosophes obvious, tous se racontent en chœur, grande clameur d’un
 savoir qui semble parfois ma foi bien utile ou juste captivant. mais non la
 majorité du temps à vrai dire l’impression de se faire prendre pour des cons.
 Jolie est rousse mais ses cheveux paraissent d’un blond terne sous les néons.
 Les cheveux juste ras des épaules, bien droits lui cachent la moitié du
 visage, les yeux disparaissent en sourire lorsqu’elle voit des gens qu’elle
 aime, le reste du temps le regard froid glisse caché elle embusque elle traque
 la vermine on lui a apprit dans le dos ça se fait pas on parle en bien et on
 descend pas on ne mine pas pour se remonter tout le monde finit juste par se
 caler c’est évident mais d’autres rush comprennent pas comme Naomi la bitch
 qui se trouve toujours un soufre douleur ou l’autre bellâtre gossant qui lui
 jette tous les jours une un commentaire stupide, Jolie ne lui en veut pas. Il
 apprendra, à ses dépens, il faut apprendre un jour à se faire aimer autrement
 qu’en narguant et bousculant, s’pas grave on s’en fout de ces imbéciles. dans
 la vie il y mieux de toute façon, par exemple Zoé, qu’elle pensait absente
 aujourd’hui parce que malade du fond du couloir elle marche vers son casier.\\

Ce qui est bien avec Zoé c’est quelle partage tout, pas juste les cigarettes
les pointes de pizz ou les cartes pokémon lorsqu’elles étaient plus jeune
(sorry les billes c’était en France dans les années 40) mais l’humour aussi.
Elle apporte un cynisme jovial dans la journée scolaire maussade. Les mots pour
se moquer les yeux pour dire que c’est pas méchant.  Une tête sur les épaules,
on se dirait qu’elle avait surement vécu quelque chose de très triste plus jeune
à être aussi mature. Les fatiguants les gossants s’approchaient pas trop d’elle.
Zoé s’accote sur le casier de Julie et pousse un long soupir qui veut plus dire
grand-chose, c’est des ados après tout, la tête de biais pour laisser tomber sa
pluie de cheveux noirs charbons mordillés jusqu'à ses pouces bien entortillés
autour des sangles de son sac à dos style surplus d'armée

\clearpage

--- Zo t’as tu fait le devoir d’Anglais?  \\
--- non je pensais que c'était toi qui allait le faire cette fois\\
--- fuck je pense quon est dans la merde cette fois pour de bon\\

la cloche sonne\\

--- bon c'est l'heure d'y aller, on va se faire ramasser par la vieille cette fois\\
--- ouais déjà qu'elle est pas charmante quand on est les fait ses traductions à la
con cette chippie de prof d'anglais

Elles quittent le halle d'entrée à la hâte.

\clearpage

\subsection{cours danglais}

\clearpage


\subsection{origine}

\clearpage

Deux heures et demie plus tard Zoé range ses cahiers de classe pour l’avant midi
dans son sac à dos \\
Elle claque la porte et commence a marcher.\\
- En-voèye on va être en retard encore pour le cours de gym, moi je gosse pas
avec une baguette de badminton une minute de plus qu’il faut; le Jocelyn va
nous donner des exercices de plus\\\
- "Punition positive "\\
- "La fonction est d’améliorer "\\
- "On amène l’étudiant à aller à son propre potentiel "\\
- "C’est inspirant comme institution"\\
- "Fuck yé weird ce gars la"\\


Elles se sont rencontrées comme membres de la même chorale au début du
primaire. Bon elles n’étaient plus insérables comme avant depuis quelques
années déjà, l’adolescence l’identité, etc. Zoé est devenue plus rough sur
les bords, aimait provoquer et foutre la marde ; Jolie se voulait ouverte
d’esprit et jeune et aventureuse mais trouvait tout ceci un peu trop obvious et
juvénile, toute cette révolte, l’ex-centrisme manifeste.\\

\clearpage

\subsection{musique}

Jolie et Zoé ont la chance de se faire créditer un cours pour leurs pratiques
de band; avec Mme. Ashkenazy comme prof et superviseuse, parce qu'évidemment
il faut s'assurer qu'elles \textit{méritent} leurs crédits.
Mme ashkenaz est une musicienne tchèque d’une
quarantaine d’année, toujours habillée de corduroy. Aigre mais sympathique,
rigoureuse mais enjouée tout de même. Jolie rejoint Zoé dans la salle de
musique son enveloppe de guitare à l’épaule. Elles vont s’asseoir à l’une des
tables et préparent leur gear ; branchent les fils, allument les amplis,
ajustent le tone. Jolie grattouille des cordes en forme d’accords lorsqu’elle
chante mais c’est Zoé qui supporte vraiment la fondation harmonique à la
guitare et saupoudre le tout de fioritures mélodiques. Jolie chante les
chansons qu’elle écrit, retravaille et compresse depuis maintenant quelques
années Elles jamment un peu pour se détendre et se délier les doigts. Mme
Ashkenazy se réchauffe à la batterie, effectue quelques manœuvres, des exercices
techniques de coordination et d’étirements.\\

Elle frappe fort : TCHAK TCHAK CRAK CHOMP TCHAK.

Ça fait un vacarme mais la salle de cours est en fait dans une rallonge du
CEGEP, un peu en retrait et isolée – et froide et mal foutue – c’est d’ailleurs
pour ça que l’administration l’a proposée (on est pas imbécile aussi bien
s’arranger pour que le moins de monde soit dérangé, toute façon personne n’en
veut de ce local plein asbestoses et humide) proposée à Mme Ashkenazy pour
l’implémentation du nouveau cours à option : “création d’un band sous la
supervision d’un professeur”.\\

Mme Ashkenazy n’est pas une réactionnaire, elle accueille autant le folk jazz
que le punk progressiste, les chansons ont parfois un certain air de scandale
et pas de souci. Mais on ne perd pas son temps, lorsqu’on arrive à la pratique,
on est prêt on a fait les lectures ; nouvelles charts, pages de manuels
techniques, essais sur l’art, etc. Le band “vieux techs” commence à être bien
rodé et les membres n’ont qu’à s’échanger quelques brefs mots, un signe ou deux
et elles commençaient la pratique. \\

\clearpage

\subsection{pause eau}

Jolie et Zoé, un café à la main sur un banc de parc qui fait face au fleuve
proche du bas de la ville. Agathe arrive, la grande mince, chef de l'équipe
de volleyball, mais elle fait ça pour avoir sa bourse pour décrisser du coin
au plus sacrant tout le monde le sait bien, et un peu pour sa mère, qui
l'accompagnait à chaque pratique en char depuis qu'elle avait 8 ans. Une punk
en pastel, elle vit mollement sa crise d'adolescence, c'est plus une crise
de condescendance. mais très raffinné, seules les personnes qui passent
quelques années avec elles peuvent s'apercevoir de l'ironie dans sa voix
lorsqu'elle répond avec enthousiasme  aux directives d'un professeur gossant
au sourire douteux où à la directrice, cette dernière très fière de son
championnat de volleyball.\\[1ex]

-- salut les girlz\\
-- allo Gate, ça a duré plus long que prévu votre pratique\\
-- Ouais ostie parle moi en pas, jta boute, depuis qu'on a gagné
l'année passé tout le monde est sur notre cas, le mien en particulier\\
-- Combien déjà McGill te donne pour rentrer sur leur team \\
-- Ben ils me payent mes frais de scolarité\\
-- Ouais mais le CEGEP c'est pas genre 75\$\\
-- Non mais à McGill, et je suis sûre de rentrer en archi\\
-- Hmm makes more sense, mais quand même, moi faudrait me payer
cher en Ta\_bar\_nak pour jouer en tit shorts proche de toutes ces vieux caliss
qui disent venir pour "encourager la région". \\
-- ouache esti, veux tu ben pas me mettre des images dans la tête Zo,
anyways, \ldots \\[1ex]
Agathe se retourne pour faire face à la berge, se penche et ramasse une
vieille chaise de toile qu'elle laisse trainée là depuis quelques années,
parce que trois sur un banc, c'est malaisant, il faut que le milieu recule
ou que les deux extrémités se penchent, une vieille chaise de toile donc,
verte forêt, après l'avoir légèrement tapotée et secoué pour enlever les débris
de terre elle s'affale dedans et sort son six-pack




--tit shorts à part vous les êtes là depuis combien de temps vous
autres\\
-- genre\ldots, depuis 2heure trente à peu près, Jo à quelle
heure est-ce qu'on a fini notre pratique\\

Jolie n'entend pas l'interpellation de sa camarade, elle rêvasse en sirotant
son café depuis qu'Agathe est arrivée, même un peu avant pour être honnête,
probablement 5 minutes après que Zoé ai commencé son rant sur la nullité
de la musique quebz qui passe à la radio. Elle a les sourcils légèrement
froncés, on dirait qu'elle regarde très loin mais en fait ses yeux sont perdus
dans la marée qui récède, les mains dans les poches, légèrement crispée, le
vieux k-way qui protège de la brise, une rumination quelconque.
C'est aussi un lieu comme un autre qu'elles ont adoptées pour se rencontrer,
tergiverser un peu entre cafés ou bières. Ce petit parc à l'aube de la berge
fait partie des derniers efforts de la mairies pour rendre l'urbanisme quelque
peu plus moderne. Une piste cyclable longe le rivage proche du
centre-ville, à certains intervales, décorée de quelques bancs de parcs et
de grosses chaises et tables étranges en palletes de bois recyclés. Les bancs
de parc sont accouplés à des lampadaires aux lumières intelligentes aux ampoules
LED, c'est un jaune opaque et mat qui rayonne le soir et virovolte contre les
arêtes de l'eau et les épinettes parsemés derrière la piste cyclable pour
écorcher les vents un peu trop vivants. \\

Ce n'est pas rare que Jolie ait ces moments d'absence, on pourrait dire qu'elle
arbore un léger TDH si on ne distinguât pas la concentration et la capacité
de se faire une sieste de l'intérieur.

---Jo?\\
---Aloo décroche?\\
---hmmm\ldots?\\
---Fak est-ce que tu viens au Bunker demain soir finalement?\\
---Je pense que oui, vous avez trouvé un lift?\\
---Gate prend le char de son frère\\
---Le pickup?\\
---Ouais\\
---Mais il a pa genre juste deux places, on était supposés ammener
Andrée aussi\\
---pas de stress, j'ai des petits coussins pour mettre dans la boîte, même
un petit cooler avec des bières pour la route\\
---Et ta mère est chill avec ça?\\
---Bof ma mère à partir de 21h elle est accotée sur le xanax et le ballon de
  rouge en train d'écouter Télé-Québec, elle remaquera même pas qu'on est
trop pour fitter dans le char. Pis toi Jo t'as récupéré tes affaires de camping
chez ta mère?\\
---Non mais ça va j'ai un sleeping bag et une bâche j'vais m'arranger.\\
---hmmm t'arranger enh? alors c'est qui que tu pensais te pogner?\\
---honnêtement, j'ai spotté un shed à quelques minutes de marche
la dernière fois donc si il y a pas de place dans ta tente,
admettons que gate soit \textit{par hasard, pas} en train
de se pogner Doménico
---eille comment t'as entend\ldots
---peut-importe,  décanter la concoction toute seule en mode boudhiste dans le shed sur le bord de l'eau\\
---avec ce qu'il va y avoir dans la potion magique je suis pas sur
que tu veuilles vraiment t'égarer dans les bois comme une petite
Hansel
---Gretel\\
---Quoi?\\
---Gretel c'est la fille, Hansel c'est son frère\\
---Whatever\\
---\ldots\\
---\ldots\\
--- fak domenico enh?\\
---Si tu le dis à mon frère tu vas manger ta gaspacho chaude fille\\
---Wow\\
---$\backslash$  o/
\clearpage


\section*{une Vue}

Jolie écrit des paroles de chanson dans son garage qui fait office de
chambre/tanière en regardant le fleuve qui se verse plus loin par la porte de
garage vitrée. Il a été aménagé pour elle il y a quelques années. C’est une
petite bâtisse de bois détachée de la maison par quelques dizaines de mètres.
Le terrain de M. Paul Diez est en pente à flanc de montagne, flanc de butte
pour être plus précis mais c’est assez ça fait que le soleil perce et l’on voit
ien la berge qui se reflète.  C’est probablement dans ces moments qu’elle est
le plus productive, de 7pm à 3 heures du matin environ. Elle a  soupé et peut
s’installer tranquillement dans le garage. Jolie s’en ai fait un nid avec un
grand tapis un vieux sofa et des disques qui traînent un peu épars. C’est
relaxant comme endroit, du Valium en pin blanc. La soirée est d’autant plus
productive si c’est l’été et une grosse pluie vient barboter sur l'eau au
loin.\\

Elle était un peu émèchée en rentrant du parc, quelques bières d'après-midi
avait suffit. Paul ne s'apercevait pas de ses choses là même s'il se levait de
son fauteuil; ce qu'il faisait de moins en moins. Jolie se fit donc une théière
de thé noir, n'aimant pas être trop vaseuese, elle rêvassait déjà bien assez.
Le garage est muni d'une mini cuisine de camper ainsi que d'une toilette, il
faisait donc office d'appartement temporaire avant de finalement pouvoir
s'eclipser, on ne sait où, ce n'est pas trop important, tant que ce soit
ailleurs. Le plus important, après le climat (svp plus habitable) serait la
distance. mais la distance c'est difficile, ça prend d'autres langues, des
avions, de l'énergie. Peut-être New-York, mais on se résignera bien sûr pour
Montréal, pas Québec en tout cas. Jolie sait que des ses camarades la majorité
opteront pour cette dernière lors du saut à l'université , on irait plus loin
en restant chez soi que d'aller là-bas.



\clearpage

\section*{océan}

L’institut des sciences de la mer de Rimouski est un centre de recherche
affilié à l’UQAR, on y étudie tout ce qui a lieu aux grandes étendues d’eau.
Bien entendu on pense surtout au golfe du Saint-Laurent. Les océanographes qui
y travaillent se déclinent en plusieurs profils ; géologues marins,
biologistes, on étudie la géophysique des courants et le plancton et ses effets
sur la faune.  C’est donc diversifié comme milieu, surtout depuis les dernières
initiatives du gouvernement qui ont pour but d’attirer les immigrants en
région. Paul Diez est chercheur en dynamique des courants thermos-salins.
Exposition sommaire du phénomène : l’eau chaude des tropiques se déplace vers
les pôles puis se refroidit, elle devient plus dense elle descend vers les
profondeurs, la salinité la rend plus lourde ; le plancher océanique est glacé
et salé. C’est à quelques milliers de profondeurs que la pression est assez
forte pour permettre à plus de sel de se dissoudre dans l’eau.\\

L’eau remigre par la suite vers l’équateur où elle se réchauffe et remonte,
l’agencement du tout produit les grands courants océaniques. En bas, dans l’eau
froide et noire ce pourrait être effrayant, avec ces poissons étranges tout
droit issus du Jurassique on dirait. Ces parcours de milliers de kilomètres
autour du globe fascinent Paul Diez, surtout la couche profonde de l’océan ;
l’abîme. Avec ces drôles de poissons, ils sont mignons après tout, et ils ne
veulent pas vraiment de mal à personne. Ils ont l’air plutôt paisibles ces
petits monstres laids.\\

Paul n'aime pas beaucoup les gens, à quelques exceptions près. Il s'imagine un
Cousteau détennant plus de moyens techniques malgré le financement plutôt
dérisoire que lui accorde le gouvernement ces derniers temps. Il a nommé son
bateau de recherche le Nordique, dans un espoir piqué d'arracher quelques
sympathies à ces philistins de la capitale qui s'épanchent encore en
mélancholiede leur désormais disparue équipe de hockey qui portait ce nom.


Paul pilote de chez lui un petit sous marin télécommandé. Il se promène ainsi à
des kilomètres de profondeurs dans le confort de son bon fauteuil mou. Parfois
il va physiquement dans un plus gros seaexplorer avec des bons sièges et des
biscottes mais il coûte cher à l’université. L’administration voit toutes ses
promenades scientifiques d’un œil sceptique. Certains d’entre eux sont un peu
morons faut le dire.  \\


Il ne le sait pas mais il glisse tranquillement vers les bas-fonds de ce monde,
maintenant la cinquantaine; il se sent attirer par les abimes. Ils les trouvent
ennivrantes et ne peut s'empecher d'y voir des motifs, des pattern, une sorte de
prédestination géologique. Il arpente grâce au sonar de son sous-marin les
récifs, les montagnes et les fentes sous-marines. Sur les forum internet aussi
il passe de plus en plus de temps. Il y en a des dizaines où s'agglutinnent les
amateurs océanographes.


\begin{comment}
Certains, les noumeniens, pensent que le relief
sous-marin est signe d'une ancienne civilization qui aurait terraformé la terre,
c'est à dire modifier son climat et ses fonctions essentiellement géologiques
pour la rendre plus habitable.  On ne parle pas d'Atlantis ou de Troie mais de
quelque chose de beaucoup plus grand, et inquiétant. Selon Alexandre Laffarie,
un des modérateur du forum phare de la théorie, ce n'est qu'en découvrant les
rites de cette civilization pré cro magnon que l'on pourrait éviter l'apocalypse
environnementale. Bipèdes ces anciens, pourquoi pas? Deux yeux, peut-être trois,
d'où la théorie universelle du troisième oeil chez les nouveaux paiens, peut-être.
Peut importe pour les ``Laffarites''; l'important est de retracer leur influence
sur la terre de façon concrète. Sceptique Paul, bien sur, c'est un scientifique après
tout; c'est aussi pour cette raison qu'il n'écarte aucune hypothèse.
Une des théorie des Laffarites est qu'il y a deux grandes civilization qui se sont jouées
la guerre climatique, une du haut, au mont blanc à peu près, et une perdue maintenant
dans les failles sismiques.

\end{comment}

\clearpage


\subsection{hippie}

La mère de Julie habite à quelques dizaines de kilomètres de la ville. Sa fille
ne comprend pas encore très bien qu’est-ce qu’elle fait pour gagner sa vie au
fait. C’est un mélange bizarre de job, elle est boulangère à ses heures,
conseillère de ville à d’autres, on a eu ouï dire qu’elle a passée son barreau
autrefois pourtant elle passe plus de temps à contempler et nourrir ses chèvres
qu’a lire les journaux, si elle lit c’est de la poésie, un peu de Tchèque et du
français bien entendu mais aussi de l’américain et elle s’essaie récemment au
portugais ce qu’elle essaie de transmettre à sa fille. “T’aimes le jazz et la
samba, c’est beau la bossa, tu pourrais chanter des balades brésiliennes?”\\
--- Maman je vais pas apprendre une langue si vous me donnez pas les moyens
d'aller la pratiquer. Il faut s'immerger dans une langue, des cours ne suffisent
pas tu le sais très bien.\\
---- Oh tu parles toujours d'argent ma fille mais tu vas voir ça ne fait pas le bonheur,
il y a des choses beaucoup plus importantes. Tu ne peux t'imaginer ce que ça fait à ton cerveau
d'apprendre, de maitriser vraiment une autre langue.\\
--- Ça fait peut-être pas le bonheur mais ça fait les billets d'avion en tout cas
\clearpage


\section{bunker}


\section{marbre}

Les couleurs se versent dans leur tiédeurs ternes et l'âme de Cédric se complait
en épithètes chialeux. Le café est trop lent, il se déploie dans la tasse, comme
une routine de yogi au sourire imbécile, mielleux et perdu mais avec quelque
chose qui cloche derrière, une paix intérieure lactée et donc trouble. La
méditation n'est pas pour celui-ci, il manque de flexibilité et ne peut dont pas
s'assoir convenablement les jambes pliées. Et méditer sur une chaise, c'est con
tout de même, on dirait qu'un principe essentiel est ainsi transgressé. Et des
principes ancestraux, il en a déjà transgressés assez ces derniers temps. Dans
ce genre de mood il faut pas rester sur place, on s'active, on va faire du
sport, une bonne course dynamique pour se brasser les os et ensuite hop la
douche chaude et puis les étirements et un bon petit poisson grillé, légumes
vapeur le tout couronné d'un bon film, quelque chose de réconfortant. \\--- ---
ou l'on fume. --- L'on fume si la morosité cynique est cause révolutionnaire; la
fuite du cliché aboutissant toujours et inévitablement en cliché, en clope et
autres symboles phalliques. Mais tout de même, après tout, il faut bien meubler
sa jeunesse. \\

Et d'ailleurs là où Cédric se trouvait, les meubles ne sont pas ce qui manque.
Ça alterne entre le contemporain lisse, le canapé ancien-régime, la bay window
entre deux vases chinois, on a droit à du granit, beaucoup de granit, et un bois
que l'on pourrait qualifier de japonais ; le rouge à lèvre recouvre
approximativement 30\% des lèvres avec goût ce qui est un ratio qui fonctionne
bien et ça indique à qui sont les drinks selon la teinte; ce qui permet de
remarquer le verre orphelin de Gallifée et de lui porter alors qu'elle contemple
paisiblement la rue McGill deux étages plus bas une cigarette à la main la
fenêtre légèrement ouverte, la fumée qui s'égare vers les bassins au bout du
Vieux-Port. \\

Le granit les talons les grands verres, très grands verres à vin, tout est
brillant et cristallin, avec de légères notes complémentaires de soyeux et de
velour, la pluie est légère et sophistiquée en glissant sur les grandes
fenêtres:

Cédric essaie
de s'extirper de sa bulle de poête cynique par le geste; il s'empare du
verre de Gallifée et essaie de se faufiler au travers de la piste de danse
improvisée, où les gens tournent et tournent et les grands talons font
tac-tac-tac et les grands verres cling cling, il bredouille un peu, aimerait
être plus souple dans le mouvement du corps, regrette de ne pas avoir appris
une danse sociale, la salsa problement, lorsqu'il était en Amérique Latine
avant d'entamer les études supérieures, il aurait peut-être eu le sang un
peu plus convivial. Il aurait dû être comme David et accepter la vie telle
qu'elle lui a été présentée au lieu de se morfondre en aphorismes à deux
piasses.\\

\emph{Un cynisme comme une peau de lion pour cacher un amour fragile.}\\

Profitons des quelques instants où Cédric s'avance le verre de Gallifée à la
main vers la fenêtre où cette dernière se berce au gré du vent d'automne pour
faire un topo rapide.

David est en train d'emménager avec Gallifée qui est toujours aussi
empathique et chaleureuse dans un condo à Villeray grâce à son salaire de
consultant en \textit{art-investment}, effleurer suptilement la hanche de
Gallifée, amicalement bien sur, (pendant que son copain Dave raconte une
vieille histoire d'universitaire à Joe histoire qui comprends une auberge de
jeunesse, un bateau, et une omellette, 3 batons de dynamites, quelques
cigares et un tigre asiatique et drogue, à risque de paraître vulgaire,
\emph{évidemment} : drogue) et tirer un sourire peut-être un peu trop gras,
mais il n'y a pas réflexion, il s'agit de réaction

s rapides. \\

Tout ceci est confus et ça ne se choisit pas les sentiments, ni ceux bien
tendres envers Gallifée ou ceux d'envie face à la situation de David. Ce
genre de comportements ou de sentiments n'ont pas leur place au sein
d'amitiés profondes qui ont l'âge d'un très vieux chien, quoique disons le,
soyons \emph{honnêtes}, Gallifée est très, très jolie\\


\begin{center}\noindent\rule{0.5\textwidth}{0.4pt}\end{center}


Le café finit par couler, une fois la toast beurrée le matin peut tranquillement
se résorber. On échange quelques bières dans un bar quelconque car on est samedi
après tout et on se ramasse par quelque mécanisme obscur dans un grand immeuble
vitré au vieux-port de Montréal, entre deux galleries trop chères qui vendent
plus du design graphique commercial léché que de l'art, que l'on se retrouve à
rigoler avec des petits regards admiratifs en coin ce qui est quelque peu
étrange d'ailleurs parce que David et Gallifée sont habitués à l'endroit, pas
précisément celui-ci mais son essence, son zeitgeist. Mais on ne sort pas en
ménage à trois, cela ne se fait pas, il faut comparses, bonhommie, du léger, des
personnages secondaires à notre vie qui ont des catch phrase et ajoutent la
bonne teneure de rocambolesque, il faut \emph{symétrie} donc il y a aussi Jean
qui est ingénieur et fait le tour du monde, il sort d'où on sait pu trop, la
Zambie, toujours la Zambie et la Malaysie surtout d'où il revient avec ses
histoires abracadabrantes, une légère barbe hirsute, de nouvelles normes
culturelles et une nouvelle personnalité qui vient se graffer sur ce qu'était
Jean pré-nouveau voyage qui change toujours mais toujours grand et blond et
blanc, en fait tant qu'à y être n'oublions pas d'appeler Joe pour qu'il se
joigne à l'excursion vers le party d'amis d'amis d'amis recursifs, Joe et ses
lunettes rondes et son humour décapant, son charisme de dents tachées démontré
lors de la marche du métro vers l'édifice; il prend la peine de s'arrêter à
chaque sortie de bar pour s'introduire dans chaque discussion avec quelque
présence féminine pour en échapper un sobriquet un sourire lorsqu'il raconte une
anecdote rapide ou pousse un compliment, dents qui n'affectent pas son charisme
car il peut se le permettre avec ses cheveux gras et lisse, ses yeux sombres et
son teint olive, ses larges poignets ses yeux olive et son regard ombrageux, son
je-men-foutisme maintenant garni d'un concluant salaire à la radio de
Radio-Canada, d'ailleurs il ne se dirige pas vers les groupes de fumeurs que
pour cruiser pendant que ses amis l'attended en sirotant une bière à la
bouteille, il en profite aussi pour discuter de sujets épars, il en maîtrise
beaucoup grâce à son boulot, toujours en train de commenter tout.\\

\clearpage

Donc on monte un ascenseur au vieux-port un ascenseur qui fait zouuu tout en
douceur avec un cockpit comme si l'on voyageait dans un tube pneumatique et on
se taquine un peu, l'atmosphère est bien détendue, on est \emph{ben cocktail}.
Ça se remarque, on se dit quand même; entre deux feintes de boxes avec Cédric
Joe craque le mirroir qui lui fait dos sur quoi la joie et la désapprobation
sont totales (car le masculin, totaux, si laid) : "Eille Joe à soir casse pas
toute caliss" --- "M'en criss on Turnn Up\footnote{Vire fous, on fait le gros
  party, la teuf quoi} a soir less go" "Joe\ldots J-J, tout-doux" --- "ouais
d'accord Quoii D'AUtres". Donc on monte dans ce tube et ça fait zouuu et on
giggle entre quelques gorgées partagées de vin blanc à la bouteille. Et l'on
cogne entre deux simagrées à cette grande porte lisse et pleine. On entre dans
ce loft mezzanine dont les deux étages donnent sur une immense fenêtre qui elle
donne sur le centre-ville illuminé et le fleuve qui s'allonge. Bien évidemment
il y a du trap, un mobilier de jeunesse flétrie--disons fin vingtaine à fin
trentaine--riche, bon rien de dynastique mais tout de même, en 2018, le mobilier
d'une telle cohorte \emph{nécessite} le trap.
\footnote{Le trap est un style musical qui a ses origines dans le hip-hop du sud
  des états-unis. Il est marqué par de très rapides coups de snare en triplettes
  sur de larges basses lines qui ondulent sous le rythme de gros gras kick-drum.
  Le tout est garnit alors de \textit{mumble rap}, un style de rap où l'artiste
  déploie paresseusement ses rhymes, lorsqu'il y en a, avec l'accent d'un
  ivrogne sur la codéine, le rythme encore en triplettes:
  tatata-tatata-tatata-TA. Nous pourrions qualifier ce dernier style d'une série
  de dactyles punchés à la fin par un anapeste moderne}\\

Le loft est situé au dernier étage d'un nouvel immeuble, les planchers de granit peut-être, on admire
le tout en se délaissant de son imperméable et en enlevant ses botillons mais
quelqu'un nous enfarge: Jean est ben trop high pour délacer ses souliers polis
ou pour avoir une quelconque appréciation esthétique soutenue qu'il se trémousse
déjà en se faisant aller les bras vers la partie plus sombre de l'endroit où le
dance floor a été méticuleusement déposé, et Joe, Joe cherche déjà les verres et
n'en a rien à foutre vraiment des bâtisses, il cherche des verres surtout pour
se chercher un verre parce que la bière ça fait pas la job et il a
judicieusement ammené un fiable 26oz de Jim Bean\\


On est dans la cuisine, on prend place, se cherche un verre, se présente
aux divers convives qui étaient déjà présents, certains pour un verre d'eau
d'autres pour fumer sous la hoote, ou encore, comme c'est le cas de Salomé
simplement pour s'éloigner de la fête parce que déjà à cette heure pas si
tardive  ça se tortille, ça fait de la grosse poudre, ça s'ostine sur la
prochaine toune, il y a à ce que l'on peut comprendre déjà eu tout
un combat de masculinité toxique, pas aux poings mais un est parti en claquant
la porte, une histoire de poker ou d'ex on ne sait plus.\\

Alors Cédric décide d'arpenter les lieux et se déplace vers les escaliers en
évitant des conversations sur la vie, l'amour et la crise financière, les
danseurs un peu trop enjoués et finalement il peut faire l'ascension du
colimaçon en bois, celui-ci nettement québécois, du frêne recyclé on dirait, et
il arrive à un cercle de petites conversations sur les fauteuils rouges amples
mais angulaires joliment installés en ménage à trois sur le bord de la rampe. Il
faut socialiser au final, on ne reste pas entre petites cliques comme de gros
quebz salles à un party, on mingle, \emph{caliss}. On fait des rencontres
inopinées avec, évidemment, la vue majestueuse sur la deuxième moitié en hauteur
de la bay window, cette lumière colorée à travers les échancrures des grands
luminaires abstrait de glissants d'étincelles.
\\

[Note de l'auteur : dialogue émotif à ajouter]\\[1em]

 sa gauche il y a une \emph{salle à poud}, la chambre en temps normal destinée
aux vacanciers américains ou français qui déboursent quelques centaines de
dollars par nuit pour l'escapade et on rentre dans cette pièce et en fait il y a
un miroir bien positionné, la vitre vers le haut, un miroir sans cadre, pour
gratouiller tout ce qui reste sans que ça coince dans les craques, scratch
scratch l'âme de rasoir et évidemment, lorsqu'on s'en fait proposer une tite
ligne, et qu'on est là pour relaxer, et que c'est un nom de la politique bien
connu maintenant, connu pour ses opinions plutôt radicales gauchistes, qui vous
proposent la dite tite ligne, alors on dit mais oui en fait allons-y.\\

Alors Cédric prend place dans le cercle ou plutôt rectangle courbé de chaises en
aluminium et fait un signe de tête et un gentil "Salut". D'ailleur juste à côté
on retrouve Joe qui roucoule comme un perroquet et fait des becs dans le coup à
une animatrice de variété autrefois connue qui a d'ailleur disparu plutôt
brusquement de la sphère médiatique Québécoise, petit fait divers intéressant
bien vite résolu par l'animatrice entre deux sniffées, elle est \emph{en thèse}
, elle en avait marre des médias et de la superficialité; elle est retournée aux
études comme elle l'explique en ce moment, en \emph{thèse} sur le poète
Brézilien Carlos Drummond Andrade et sa démarche formelle face à la langue
populaire, \\ on a plus les animatrices de variété qu'on avait\ldots


Les petites heures approchent et il se retourne à contempler la vie et Salomé,
la jeune femme avocate sincère et spirituelle qui lui fait face dans la cuisine
entre le fridge et le comptoir auquel elle est indolemment accotée. Il voudrait
lui contempler les bas-fonds de l'âme et s'y plonger, mais les heures sont
petites, ses yeux sont vitreux, la musique se fait longue et plate. Il fixe un
ustensile, n'écoute rien, ni ce qu'elle dit ni le bruit de fond constant ni les
paroles du rapper \textit{Lil-Mickey-Royce}. Il lance quelques regards autour de
lui pour constater une étrange apathie, et il faudrait percer l'air et rejoindre
Salomé ou quelqu'un quelque chose. Regards croisés, une discussion authentique?
On se voit s'ouvrir à cette belle étrangère qui nous expose un intéressant
dilemme éthique dans le droit international. Faire une vraie rencontre et
prendre rendez-vous, pour une marche sur le Mont-Royal, avec un chien, c'est
l'automne, c'est coloré. Mais elle parle dans le néant, il se retourne, plonge
sa main gauche dans un gros bol de cheetos et pendant qu'elle élabore sur la
constitutionnalité post-moderne; il se liche un à un, lentement, chaque doigt de
la main gauche. \\


Joe est probablement déjà rentré avec quelqu'un(e) il ne pourra donc pas
remonter le moral à Cédric avec quelques jokes de mononc bien tournées et des
gesticulations (c'est sa seule utilité)\\

Cédric s'avance le verre de vin à la main, verre toujours
taché du rouge à lèvres sobres de Gallifée, en boit un grand trait et le dépose sur
une corniche car la fenêtre est ouverte et donne sur un faux balcon. Jean et Joe
cassent quelque chose de vitré en dansant, si on peut appeler cela de
la danse à cette heure-ci, c'est plutôt un rassemblement amateur de danseurs
du ventre. David vient rejoindre Cédric à la fenêtre, lui tend une bière.
Les deux prennent une gorgé, haussent les épaules. Le premier fait à l'autre un
signe de tête. Ils sortent et descendent les escaliers.\\

Une fois sur le trottoir de la grande rue McGill avec ses nouveaux lampadaires
chics et sa belle asphalte large et ondulée et les commerces de luxe ils se
dirigent lentement vers le port en allumant un joint.\\

Arrivé à la promenade derrière à la piste cyclable ils s'avancent vers la
fin d'un pier, comme une presqu'île pittoresque.
Ils prennent place à un
banc, râlent contre les conneries de la vie, quelques vicissitudes partagées
malgré leurs parcours divergents. Ouvrent chacun une cannette de Old Milwaukee,
par nostalgie de l'adolescence, David humecte la colle d'un autre joint alors que
son ami s'essoufle d'un soupir mélancolique mais paisible. \\
-- Pis Dave, tu penses tu que ça va ressembler à ça votre loft une fois retaper pis toute\\
-- Non dude, voyons, j'ai tu vraiment l'air d'un gars qui plaque des
-- reproductions de Jackson Pollock partout \\
-- Ben non Comon jte niaise\\
-- Je sais mais ça hit fort quand même de voir du monde de même avec qui
t'as jamais eu tant que ça en commun et te dire, ben oui ce serait logique,
ce serait moi dans pas long tout ça \textelp{} Pis toi, t'a fini ta maîtrise tu
vas tu au Doc?\\
-- Je sais pas trop encore, ça pu l'air trop pertinent, j'ai l'impression
de juste ingérer des bits d'informations, style oie à fois gras\\
-- Je t'entends, même vibe pour moi quand j'ai fini par finir l'école\\

\textelp{} et au fait, maintenant que j'y pense, pour votre appart là, vous
avez pas aussi commandé le même genre de comptoir contemporain en granit
messemble\\
-- C'est pas du granit, \emph{criss}, c'est du \emph{marbre}\\

Cédric humecte maintenant le joint qui lui est repassé en le tournant entre son
pouce et son index, déposant la salive avec son auriculaire à l'extrémité du
cherry, il s'émouvoit encore un peu du paysage, urbain mais intime quand
même\ldots quelques rares passants, la lumière du port, une eau trouble et
miroitante.\\[1ex]
Il décide qu'il est maintenant impératif de séduire Gallifée; préférablement sur un comptoir.
\clearpage


\section{mirroirs}

Nous sommes en Février 2018; un mois passé le party dans le loft style gallerie
d'art au vieux port. Gallifee est assise sur le comptoir de marbre, bière à la
main, Cédric rêvasse, dans un fort intérieur lointain. Il a été invité à souper
par le couple bienvennant, on est à la moitié de l'hiver. Il manque désormais de
conviction face à la femme qu'il désire. Ce qui lui manque aussi, peut-être même
beaucoup c'est la tendre jeunesse de ses vingt ans ou jamais il n'imaginerait
faire un de ses bons amis cocu, maintenant la pensée lui revient à la tête de
mannière de plus en plus violente, et ce malgré le support émotionnel non
négligeable que ce joli couple bien convenu mais bien agencé aussi lui porte.
Ces contradictions ne sont pas la bienvenue dans la tête de Cédric, il cherche
cohérence avant tout. \\

--- le pire, c'est le pire de calisse de mois de l'année que s'exclame Galiffee\\
Dave acquièse en prenant une gorgée.\\
--- moi, dit Cédric, j'aime l'hiver à son plus froid, d'une façon un peu tordue mais quand même,
c'est primitif, on se caline, on se recroville ou l'on hiberne mais chose on est sur,
on est attiré par tout ce qui est foetal. Le foetus m'a toujours fasciné, comment
chaque être humain aspire a se recroviller sur soi même jusqu'a cesser d'exister.\\
--- l'attrait de l'hibernation, de la régression foetal ne m'a jamais interpellé
à proprement parler dit Dave. Moi ce que je voudrais au contraire c'est me décontracter
et m'extensionner, avaler des quantités surnaturelles de bouffe, descendre des montagnes
plus hautes que l'everest. J'ai envie de bouffer des galaxie\\
--- je pense être la seule personne pas complètement mésdadaptée ici dit Galiffée\\
Dave continue\\
--- consommer des marées entières me baigner dans une mer gigantesque mais surtout,
et c'est important, la traverser, la soumettre à mon humanité. Le monde n'est pas assez
humain je trouve. Ce n'est que par l'eucharistie modernisée que l'on pourrait peut-être
remplir notre potentiel.\\
--- laisse moi te poser une question alors, comment créer de l'intimité\\
--- je pense que je saisit\\
--- toi et fée dans cet appart, on sent la coquille\\
--- oui\\
--- Il y a une énergie, une protection, une chaleur\\
--- Les murs de brique\\
--- C'est un appart ou j'aimerais me réveiller un tiède matin de noël\\
--- les bibliothèques, avec des livres, surtout sur l'art\\
--- tous mes livres parlent d'une façon ou d'une autre de la mort,
je crois que c'est une des raisons pour laquelle mes couples ne durent pas\\


L'invitation fait suite à une rumeure dans le cercle d'ami comme quoi Cédric
n'irait pas très bien ces temps ci. Célibataire, à 26 ans, on pourrait le croire
plus enthousiaste face à la vie malgré les journées brèves et frettes que le
tabarnak. La valse des temps modernes se berce cependant au rhythme de l'anxiété
et notre protagoniste n'y fait pas exception. Il essaie de lire un bon bouquin
dans les cafés, prendre des marches emmitoufflées dans un gros kanuk qui lui
descend jusqu'aux genoux. Malgré les regards et lumières feutrées de ces
endroits qu'il arpente pour faire sa dose de visages étrangers, puits de son
imagination, son appartement morose rue Christophe Colomb lui casse les pattes
dès qu'il y retourne.\\

Ça fait contraste avec celui de ses hôtes qui est bien habité. L'appartement est
habillé de façon iconoclaste, il y a du moderne, des chaises élancées de bois et
de plastique blanc qui entourent la table à diner, derrière celle-ci, en haut de
la porte du balcon arrière quelques masques africains, après tout
pourquoi-pas.\\

Pour séduire une femme il faut lui décrire sa réflection, agir comme un
mirroir le plus parfait qui reste néanmoins déformant. Elle présente,
on représente. Beaucoup se trompent à penser que le mieux est d'afficher
une forme impressionnante, soit intellectuellement ou physiquement. C'est
la une erreur aussi grave que commune, l'amour se joue dans l'imitation
avant tout; la solitude étant un puit sans fond dont on ne se tire qu'en
affichant avec fiereté la pause selon laquelle il y aurait connaissance
partagée des hontes, défis échoués et écoeuils de la vie.\\

--- tu vois par exemple là dit David, je tape dans une assiette de pétoncles
en plein hiver, et ils sont gros, mais ce n'est pas assez. J'aimerais trouver
des pétoncles aussi gros que mon poing, et me défaire la machoire à essayer de les
engouffrer d'une shot\\
--- j'ai peur de la grandeur moi, je monte une tour de point de vue et j'ai le vertige
tout de suite dit Cédric, je crois que ce que je recherche dans mes contacts avec
le sexe féminin c'est une douce chaleur qui me rapetisse, comme des épinards que l'on fait
fondre. Tient c'est ça que j'aimerais; fondre, me fondre dans ce monde frette de langues collées\\

Galiffée déssert sans un mot les assiettes et replace à la place des anciens ustensiles
des petites fourchettes à dessert.
--- Tu veux de la camomille ou un vrai thé Ced?
--- un thé si possible, la cafféine, ça reste ma drogue préféré, d'ailleurs

Cédric se penche et ramasse son sac dans lequel il a ammené tout son matériel
à cannabis. Il se roule une immense canna à peche, histoire de digérer. C'est ce
qu'il dit à ses convives à tout du moins.\\

Le repas est fini et les assiettes sont dans l'évier, on est en mode post-communnion,
c'est l'heure de dire les vrais affaires, la bière à David traine dans ses mains, il
joue à égratigner son étiquette. Ça sent les fruits de mer, ce n'est pas la saison mais
bon, ça se congèle un pétoncle non? Cédric est allergique mais ne l'avait pas dit à ses
hotes lorsqu'il est arrivé. Une légère allergie, pas comme celle aux arrachides, rien
qui fassent gonfler la gorge. \\

--- tu sais tu peux tout nous dire à fée et moi \\
--- je sais merci je sais pas trop comment expliquer comment je me sens ces temps-ci\\
--- tu m'avais pas dit que tu avais rencontré une fille cute récemment dit Galifee du
haut du comptoir\\
--- oui Jolie qu'elle s'appelle, ça lui siet bien d'ailleurs elle est mignonne\\
--- Premier rang, fascinant, j'adore le nom, ses parents étaient hippie?
--- Non pompier et policière, respectivement\\
--- Magnifique, dit dave en quettant le joint à Cédric, franchement splendide
--- Cool babe ! c'est lfun ça, comment ça s'est arrivé, votre rencontre?\\
--- on était tous les deux à un concert, elle portait un beau chapeau, je me suis enfargé dedans\\
--- cutee\\
--- vous vous revoyez bientôt ? t'a attendu combien de temps avant de la texter,
parce que j'imagine que t'a prit son numéro, on est trop vieux pour s'ajouter comme ami
sur facebook, tu le sais ça? bien je suis content dit David.  \\

En arrière il y a du jazz qui joue, un trio piano de bill evans pas mal.
Cédric a un haut le coeur, ça se trimbale ça trémousse dans sa région ventrale.\\

--- non je n'ai pas attendu des le lendemain je l'ai appelée; j'aime pas les
jeux et connivances des nouvelles rencontres. Moi si une fille me plait, je lui
fait savoir autant que possible, sans tomber dans le pathos, évidemment\\
---- évidemment\\

Le jeux de mirroirs doit être accompli avec quelque peu de virtuosité bien
entendu; cependant un brin d'humilité ne fait de mal à personne. Peut-être
bien que cette dernière qualité est la plus importante. Pour refléter à une personne
ses désirs il faut savoir où elle s'arrête et nous commençons. L'ordre des choses
doit être clair. Leur raison immaculée.


--- mais j'ai encore l'impression de déconnecter de la réalité desfois\\
--- tu fumes tu encore du weed chaque jour dit David en lui repassant le joint qui le mettrait
à terre s'il continuait.\\
--- ouais trop je sais, c'est étrange on dirait que je vois des pattern partout ça me harcèle,
comme si j'avais fait trop de maths et dans chaque recoin et crévisses de me sens je vois des équations;
de motifs qui m'échappent mais sont quasiment à ma portée.\\
--- ils sont la mais c'est pas pour ca que tu les comprends, les pattern \\
--- Von Neumann a dit quon ne comprennait jamais les maths, on ne faisait que s'y habituer\\
--- mais toi tu a l'impression de comprendre plus de la vie mais tu n'a pas l'air de t'y habituer\\
--- exactement ma compréhension est inhabituelle, je ne vois que les extrémités de ma pensée,
c'est par la que la chaleur sort de notre corps après tout\\
---  on a parlé  à Joe il m'a dit que tu obsédais sur les cahiers perdus d'un mathématicien
ça l'inquiétait un peu\\
--- Ouais le pire c'est que je comprends rien aux maths de ce gars la c'est de la géométrie algébrique
surtout qu'il faisait. Et check ça, il a fini sa vie avec deux décennies d'hermite complet, sur les
dernière photos il a l'air d'obi wan kenobi. En tout cas c'était un anarchiste et pendant ces
deux décennies ou il vivait tout seul dans les pyrénnées il a continué à faire de la recherche,
c'est ce qui est beau des maths, on peut faire de la recherche juste avec son cerveau, et bref
un groupe d'étudiant à rammenner ses travaux à paris et ont a peine commencer à
éplucher de ces manuscrits.\\
--- ça sonne comme si tu cherchais la clé de l'univers là mon cedric\\
--- ouais je sais mais check quand je fume on dirait que je comprends plus encore\\

regards consternés échangés entre Galifee et David suivent\\

\clearpage

Cédric avait commencé ses études de mathématiques à l'université McGill à l'âge
de 19 ans à l'université McGill, finissant cet hiver sa maîtrise en
optimisation. Les travaux du grands mathématiciens Alexander Grothendieck
l'obséde même si il arrive à peine à les déchiffrer. Il faut savoir
que pour comprendre ne serait-ce qu'une introduction hative à ceux-ci il faille
maîtriser plusieurs sujets à un niveau avancé, de la topologie à l'algèbre
abstrait. Or Cédric n'a pas les capacités qu'il espérait se découvrir. Avant de
rentrer à l'université il avait été premier de classe dans tous ses cours sans
trop d'efforts, il se voyait prodige d'un monde abstrait assez jeune.\\

L'université McGill cependant avait un département de Mathématique reconnu
Mondialement et on y voyait des étudiants de partout dans le monde qui
travaillaient à devenir les meilleurs depuis bien plus longtemps que Cédric.
Malgré tous ses efforts, et sa consommation d'amphétamine disons, joyeuse, il a
mal à tenir le rhythme. Et quelque chose le tracasse de plus, il voudrait,
malgré le très grand niveau d'abstraction de ses études faire quelque chose de
concret, de palpable pour ce bas monde.\\

Anarchiste depuis ses 16 ans, ayant lu d'abord l'introduction de Daniel Guérin
sur cette philosophie de la liberté il avait ensuite dévoré les écrits de Emma
Goldman, Enrico Malatesta et puis l'histoire de la seconde ?? internationale par
James Guillaume. Il se voit construire des génératices d'énergie électrique
pour des communnautés dans le besoin ou encore des stations d'épuration des eaux
à bas prix. Et il croit que derrière toutes ces entreprises, ce serait une
bonne maîtrise des mathématiques, language de la physique et de l'univers tel
qu'il se le concevait, qui l'aiderait à designer et voir construire des machines
aidant les peuples les plus démunis à s'affranchir. Bref un optimisme surrané le
poursuivait dans toutes ses démarches.\\

Or les mathématiques sont labyrinthiques, y rentrer est plutôt facile, mais sortir
de cet édifice crystallin de la connaissance est une toute autre histoire, les couloirs
cachés se multiplient à force que l'on avance, les trappes ainsi que les pièges
ne manquent pas et une fois que l'on a gouté à la rigueur d'un théorème et d'une
démonstration formelle logique claire, il est très difficile de remonter à la
surface, puisque c'est une pyrammide qui plonge dans le monde des idées et où
les notions naturelles d'orientation telles que haut et bas s'affranchissent et
se mettent à tromper l'oeil de l'esprit en jeux de mirrors incessants.\\


Cédric avait donc choisi comme spécialisation à la maîtrise l'optimisation,
sujet qui permet de s'attaquer à des problèmes de la vraie vie mais gardait un
oeil nostalgique vers la topologie et l'algèbre abstrait qui lui avait ouvert un
monde de possibilité. De plus il se sent une affinnité particulière pour le
grand Grothendieck qui avait dit un jour que ses trois passions dans la vie
étaient les femmes, les mathématiques et l'alcool, dans cet ordre. Cédric
travaille donc d'arrache pied la nuit, un joint à la main et la boite
d'amphétamines à portée de main à comprendre ces sujets mathématiques plus
corsés que sur quoi portait sa recherche. \\

Ses études officielles quant à elles portent sur comment optimiser le
placement des turbines sous marines, qui permettent de générer de l'électricité
à partir des courants sous marins. Pour leur trouver un emplacement idéal sur le
lit d'un cours d'eau encore fallait-il optimiser une fonction qui dépend non
seulement des courants à de grandes profondeurs mais aussi par rapport à la
distance aux point de branchements qui permettent d'acheminer ce courant
générer vers les lignes terrestres. \\

Pour ce faire il a accès à un immense ordinateur qui habite le sous sol du
building Burnside Hall de McGill, seul machine dans l'université qui lui
permette d'effectuer les simulations probabilistiques et d'optimiser leurs
constats dans un temps raisonnable. Toute sa thèse était donc passer du monde
abstrait des mathématiques au monde de l'informatique car il fallait programmer
cette énorme machine, le tout étant superviser par le jeune Professeur
Holderlin.\\

Ce train de vie sans relâche avait commencé à l'automne et il continue
maintenant dans les nuits frettes que le tabarnak de Février. La routine s'étant
installer; de midi à 19 heures Cédric travaille à McGill sur son modèle de
simulation océanographique, étape la plus importante avant de décider ou
emplacer les turbines. Par la suite souper à 20h suivi d'un film avec join très
long qui finit en cane à pèche ou en bande mou par sa longueur en regardant
un film russe. Il aime particulièrement ceux ou les plans se transforment en
photographie immobile tellement l'absence de mouvement les marquait comme celui
qui ouvrait Elena de Zvyagintsev, ou la caméra capte un arbre en premier plan et
derrière un grand appartement moderne à grandes fenêtres pendant une minute
avant qu'un corbeau vienne se poser sur l'une des branches qui se plie
légèrement sous son poids. Par la suite un cachet d'amphétamine de plus il
réouvre ses livres d'algèbre abstrait et de topologie, et fume une autre
canne à pêche en s'installant devant ces derniers avec un grand café.\\



--- vous savez ce qui est le plus fou? c'est que je l'ai jamais vu, le gros
ordinateur, le mainframe comme ils disent en anglais, sur lequel je simule les
vents marées et courants, il est caché au sous-sol et je ne peux y accéder qu'en
ssh, qu'est ce que c'est ssh? alors pour comprendre il faut aussi comprendre ce
qu'est un shell, tu vois un système d'exploitation, le logiciel qui gère tout
ordinateur, aussi petit qu'il soit, il a un noyau, un coeur, un centre qui gère
tout ce qui est fondamental à son fonctionnement, comme la partie reptiellienne
du cerveau, l'accès à la mémoire et au CPU. Et autour de ce noyau, il y a dans
les systèmes UNIX, c'est à dire les seuls qui sont vraiment utiles, un shell,
une coquille. Cette coquille permet d'interragir avec le noyau. Mais évidemment
on ne parle pas encore d'environnement graphique dans ce système ou dans ma
description, on est pas rendu là. Alors ce shell comment on l'utilises, et bien
soit en mode programme, ou on écrit dans un fichier texte une série de
\emph{commandes} à effectué soit en mode interactif. On appelle cette interface
un REPL pour read eval print loop. En mode ssh, on a ainsi accès à la coquille
de l'ordi en mode repl, on envoie une ligne de commande, le shell la li, evalue
ecri a l'écran les résulat, puis loop, c'est à dire le processus recommence et
le shell attends une nouvelle commande. Donc je ne vois jamais la machine mais
j'interagis avec elle tous les jours si bien que je commence à l'affectionner un
peu, je devrais lui trouver un petit nom d'ailleurs. Pour l'instant je ne sais
comment l'appeler
--- et ça dit David, ça peut se faire avec n'importe qu'elle machine, ssh comme
t'appelles? sans jamais avoir d'interface graphique\\
--- oui précisément parce que avoir des fenêtres et tout le bataclan ça demande
plus de bande passante réseau, alors que en mode shell, ce n'est que du textes
que tu envoies soit, quelques centaines de 0 et de 1 à la fois\\
--- tu sais ce qu'il te faudrait dit David c'est une application de rencontres
de mainframe, ou de n'importes quelle grosse machine à simulation, avec des photos
des petits commentaires laissés par les utilisateurs\\
--- ah tu sais c'est pas con comme idée dit Cédric\\
--- moi je pense que t'a besoin de sortir plus dit Galifee\\
--- je suis d'accord mais pour faire quoi au juste\\
--- je sais pas moi, tu skiais pas avant, messemble ça te ferait du bien\\
--- tu sonnes comme ma mère\\
--- ahh ça c'est peut-être parce que j'ai la voix de la raison\\
--- oui non mais ça m'emmerde le ski maintenant \ldots depuis les Alpes maitenant.
D'ailleurs c'était comment votre voyage en Autriche Dave, avec Joe vous avez pas du vous
ennuyer\\
--- C'était incroyable\\

Ce que Dave ne raconte pas c'est que Joe lors d'une sortie à Chamonix s'était
fait copain copain avec des banquiers de Genève qui les avaient emmenés à
\textit{la ténébreuse}, bar de denseuse de la plus haute bourgeoisie alpine. Il
aimerait montrer à son ami les photos qu'ils ont pris en douce et sur laquelle
figure les plus belles femmes de la région, si Cédric avait pu voir la photo, il aurait
vu une certaine prénommée Jade, une québecoise de passage dans la région.

--- et cette Jolie tu vas la revoir quand dit Galiffee\\
--- cette fin de semaine on va prendre une bière, proche d'ou elle habite, dans Rosemont.\\

Comment définir l'amour; dans ce jeux d'imitation est-ce qu'on ne se perd pas un
peu, et bien justement, le vrai le grand le passionnel est défini récursivement,
la récursion, autre concept fondamental mathématique et informatique. Et bien
l'amour c'est lorsque l'on place un mirroir face à un autre. Il y a ainsi
récursion mutuelle, je te vois comme tu me vois que je te vois que tu me vois.\\

Après une bonne soirée de conversation avec ses deux amis Cédric se sent plus
groundé, il remarque qu'effectivement ses recherches sur le mathématicien
Grothendieck enfumés de THC le rende dangereusement proche d'une coupure. Il
décide donc de mettre le reste de pot dont il dispose à son appartement dans le
sac donation pour sans abri qu'il tri chaque semaine. D'habitude il ne met que
quelques vieux livres, un 10 piastres et les cannettes de bière consignées dans
le sac de plastique transparent qu'il met à la rue.

\clearpage

\section{la tenebreuse}


David et Joe s'était eclipsé de Montréal pendant les vacances du temps des
fêtes, du 27 Décembre au 7 Janvier pour être plus précis. Leur travaille très
demandant leur avait laissé cette mince fenêtre d'opportunité, ne disposant que
d'une quatorzaine de jours de congé. Ils sont tous les deux d'excellents
skieurs, ayant fait de la course durant leur jeunesse pour l'équipe du Mont
Tremblant. Cédric aussi avait été dans une équipe de descente mais c'était dans
les cantons de l'est, une autre région administrative en ce qui concerne la
fédération de compétition, il n'avait donc skier avec eux qu'a quelques reprises
en contexte de course où les skieurs portent des skin suit, sorte de tunique
moulante pour défier la friction du vent. Ils ne l'avait pas invité parce que
ces derniers menaient un train de vie tout autre que celui de Cédric et s'était
comme pris pour acquis que Cédric ne tiendrait pas le coup, financièrement
parlant. David et Joe s'étaient réservé des chambres dans l'hotel le plus
luxueux situé dans le centre même de la vallée de Chamonix qui compte plusieurs
autres petits hameaux reliés entre eux par un joli train rouge.\\

un soir ils étaient sorti dans un bar de l'hotel et avaient rencontré messieurs
Duclos et Véroche, ``clochette'' et ``vavoche'' pour les intimes, deux banquiers
suisse ou Français, dépendemment de labyrinthique situation fiscale du moment,
il vivaient à cheval sur la frontière de ces deux pays si proches mais aussi
très facile à distinguer. Par exemple quand on habite en suisse on est riche.

Et donc David et Joe qui prennent le bus de la vallée du mont blanc. Ils ont
le kit parfait complet, les masques de ski mirroir argentés, les manteaux fluo,
au cas où il y aurait urgence, les pantalons noir, pour rester un peu classe
quand même.
\\

\textit{claude simon qui descend les grands montets}\\

\textit{dialogue de finance dans le bar sur flashboys}\\





\clearpage


%été 1
\section{la théorie des nids}

\subsection{nid}

\textit{on revient dans le présent}

Pour Jolie c’est la dernière et sixième année à Montréal; elle se
trouve au même nid depuis deux étés. Trois colloques, toutes
gentilles, le grille pain est efficace, il y a une petite galerie en
avant avec un set de patio éclectique, des tas de coussins et des
chaises adirondaques. \\

C’est le début de l’été elle s’assoit sur l’un
des fauteuils, fait ses lectures en après-midi. Elle a apporté avec
elle dehors quelques volumes de poésie et des revues type
national-geographic avec des grandes photos de mammifères marins
immenses et paisibles et des chutes d’eau tropicales comme si c’était
le monde dans lequel on vivait. \\

La rue Casgrain lui fait face elle
prend une pause pour s’étirer une heure ou deux après s’être
réveillée, boit un café et fait du people watching en mangeant une
courge spaghetti. Elle range un peu les coussins, taponne le tout, un
bol de salade au couscous traîne quelque part, une dernière bouchée,
le soleil ne devrait pas tarder à s’éteindre. Depuis quatre ou cinq
mois c’est Cédric qui visite, plus jeune de quelques années, il est
mignon et gentil quelque peu naïf et anxieux, mais il séduit avec ses
yeux nuageux d’ailleurs, d’un peu plus loin.\\

Il débarque de son vélo lui glisse un sourire s’assied a terre lui
demande de raconter sa journée. Il reste de la lumière ils en
profitent pour en faire de l’ellipse le temps ça se caresse ça se
domestique, on lui donne des commandes avec des biscuits et du
chocolat les minutes grésillent comme un bruit blanc le ciel délavé
vieux jeans. La chambre est à repeindre juste les bobettes à remettre
il en met partout il se tache et elle se fout de sa gueule il n’est
pas doué. La pizza est à terre Jolie aussi, assise en lotus la bière
aux lèvres.  Ça finit dans le lit, même si l’odeur de peinture c’est
pas génial c’est l’été faut bien se gâter se faire du bien. Ils se
promènent et mordillent les draps les draps volent Jolie chante. C’est
simple et collant, ils s’endorment, couchés en croix une tête sur le
ventre de l’autre, des oreillers qui traînent. Un peu de musique, ça
se mélange au vent et au ronronnement du fridge.\\

Elle a un soupir, le chien aussi. Les deux rient, ils s’endorment.
\clearpage
Cédric est un peu pathétique lui laisse des poèmes écrits en coin de
tables à côté du matelas au sol. Il continuera à en écrireElle dort un peu encore, c’est la
sieste, ce soir elle chante dans un bar. Ça la touche malgré tout ;
elle en garde quelques un par la suite, ils la suivent dans une petite
boite en carton, par exemple :\\

Avec tes taches de rousseur, poussières de feu\\
ça éclate tu es mon camion d’aube tu\\
verse dans le large une greffe de rayons\\
jette les murs pour des clairières\\
l’herbe haute l’air sec m’exfolie\\
le creux du sourire\\
‘ s’ouvre et on se berce hier s’arrête\\
demain commence après on verra\\
peut être\\
à petits pas\\
dors sans moi t’es bien\\
tu t-loves un peu dans les draps\\


d’une journée sans fin, ça s’étire \\
d’être de même, comme avars de paix \\
j’hallucine l’écrin je le sais\\
le vrai se condense pas\\
sur des brillants de douceur\\
Il faut que les vents fauchent de la scrape\\
l’amène dans les airs il faut\\
des noyaux pour que ça condense,\\
un grain de sel\\
une tache de poussière\\
tes taches de rousseur\\
\clearpage


\subsection{dialoguePostNid}
 \textit{Jolie et Cédric, à moitiés endormis dans le lit, les
draps blancs épars, légers, la dernière lueure de la journée a passé
mais il fait très beau, la nuit est claire}

Est-ce que tu m’aimerais même si je louchais\\
évidemment\\
Et si il me manquait quelques doigts\\
ça tombe sous le sens\\
mettons que j’étais amputée, qu’il me manquait les pieds ?\\
je te baiserais les moignons\\
C’est facile comme ça ?\\
Oui c’est facile\\
T’as raison; \ldots trop facile\\
(…)\\
Et si j’avais loucher quand on s’était rencontré, je t’aurais quand même fait tourner la tête ?\\
(…)\\
Si mettons, quand on s’était rencontré j'avais eu qu’un seul sourcil qui me
fendait le front\\
mais, mais tu sais bien que\\
Attend :\\
si j’avais été paraplégique ? Ou mieux ! une femme tronc, sans bras ni
jambe ? Ça T’aurais excité ? Tu aurais pensé me faire l’amour quand même quand
nos regards se sont croisés à l’orée d’une banquette sketch de bar hype\\

\ldots je t’aime\\
Oui mais avant, avant t’aurais aimé ça, un \textit{moignon} ? \\

\ldots

Raconte moi la fois où tu étais heureux\\
On était 5 amis, dans un bar, au coins de st-laurent je crois.
Je buvais une bière, on avait rit.\\
Raconte moi la fois où tu étais triste\\
J'étais tout seul chez moi et je venais de fumer
un paquet de boute en boute, Mes poumons goutaient le
et j'avais oublié la raison
\clearpage

\subsection{cyprine}

\textit{description de relation avec Jolie:
  commence en hiver, soirée de danse polaire, igloofest que cédric n'aime pas,
  mdma, le début de l'été ydillique puis passé mi Juillet le quotidien déprime
  la ``vilaine fille''}\\

Le mois d'aout est toujours collant à Montréal, pour certain ça invite
l'intimité, pour Cédric c'est un temps qui l'ammène à s'éloigner du contact
humain trop proche. Il n'affectionne pas particulièrement les parties gluantes
et mouillées de la vie; leur prefère le froid sec d'un chalet lorsqu'assi devant
un bon feu de foyer avec des bas de laine.\\

Comment décrire les sentiments de Cédric, face à Jolie, leur relation était
difficile à concevoir en premier lieu. Lorsqu'il y a fallu mettre mot sur
l'histoire Jolie eu une idée géniale, partenaires de cyprine; étant le liquide
gluant et réconfortant qui découle de ses lèvres du bas lorsque Cédric l'excite.
Le problème étant et restant toujours quant aux autres, est-ce qu'on doit se
garder à une seule partenenaire de ce type. Or Jolie croi à l'amour libre, sans
restreinte. Dans la chambre de Jolie un matin elle se lève et tire les rideaux
d'un grand mouvement large des épaules. Elle ne porte qu'un petit slip noir
et Cédric appercoit le galbe de son sein gauche et son sourrire moqueur lorsque
Jolie se retourne pour voir le soleil se jeter à sa figure. Il a légèrement mal au ventre,
plus mince que jamais il se regarde la ceinture abdominale, satisfait, et met ses
pantalons de jogging lentement en émettant un son à effort. La veille avant de
s'endormirent ils avaient discuter sur leur situation. On était maintenant en Mars
et il venait le temps de mettre des définitions, des points sur les i, bref se situer
l'un face à l'autre.\\

--- si tu en a envie un jour je n'ai pas envie que tu te retiennes pour moi\\
--- et si ça te fait mal?\\
--- Comme je t'ai dit, c'est ta vie, je veux que tu la vives pleinement, par mon amour,
je ne sais pas comment je me sentirais si tu voyais quelqu'une d'autre. Pense pas que l'idée
me réjouisse mais je préfère avoir mal que de te faire vivre une fausseté, la fidélité quand on
fantasme sur une autre est le pire mensonge selon moi.\\
--- mais j'ai pas envie de te faire mal non plus mais \ldots \\

Cédric en avait marre de ces questions auxquelles on répond en les posant (et
est-ce que je devrais en voir une autre si pourrais te faire mal). Avant de
s'endormir il lui avait caresser les cheveux en lui disant ``être libre c'est
beau, mais pouvoir te faire mal c'est un cauchemar, vient qu'on s'endorme coller
pour l'oublier. Elle l'avait trouver quelque peu insipide mais la chaleur de son
corps et son poids sur elle après l'orgasme étaient rassurant. \\

Jolie travaillait depuis quelques années au café des forêts, dans le quartier
hochelaga-maisonneuve au coin des rues ??. Cédric se méfiait du
gérant/propriétaire, Danny Sylvestre. Toutes ses employées avaient entre 19 et
24 ans, toutes séduisantes malgré elles. Il voyait les habitués les draguer sans
trop d'égare face au professionalisme qu'elles devaient garder. Parmis celles ci
il y avait son amie d'enfance Agathe qui avait quitter l'équipe de Volleyball de
McGill depuis maintenant un an. Jolie et elle travaillait en parfait synchronisme entre
la grosse machine à espresso et celle à panini.\\

Une fois l'éblouissement du soleil ayant fait son effet Cédric se leve
tranquillement. Jolie elle, était déjà sur son départ pour aller ouvrir le café
qui n'était qu'à quelque coins de rue de son appartement. Il prends son premier
café dans la petite cuisine froide de l'appartement. Il ira prendre son deuxième
au café des forets. Il ammenera aussi son sac à bandoulière rempli de bouquins,
deux romans, un dans lequel il n'arrive pas à s'immerger, et puis un Don
Delillo, toujours une valeur sure pour Cédric.\\


\clearpage



\subsection{deuxieme dialogue post nid}
Cédric et Jolie, tous deux couchés sur le lit de cette dernière

--- Tu sais que j'ai un problème d'amour dit Cédric\\

comment ça\\

et bien ca vient  du fait que je ne sais pas qui tu es vraiment, es
tu l'amas de cellules qui se reproduisent sans cesse sans jusqu'a ce quelles
soient toutes remplacées\\

je te suis\\

ou alors toutes les conversations que nous avons eues\\

ca me semble réducteur \ldots \\

moi aussi\\

en fait je ne sais comment t'aimer parce que je ne sais comment te définir
et même platon ne peut m'aider dans ce cas ci, un bipède sans plume  c'est con
tout de même\\

moi je dirais que tu me définis en m'aimant et non le contraire, en projetant
ton image sur moi, en me targuant de toutes les qualités nécéssaires à ta survie
émotionnelle\\

donc personne n'entend l'arbre tombé si il est mal aimé\\

c'est un peu ça je suis content que tu comprennes\\

\clearpage

\subsection{virtuelle}

Au début du printemps, pendant que les échanges entre Cédric et Jolie restaient
plus au moins espacés il avait rencontré Jade sur une application de rencontres.
On ne distinguait pas grand chose sur sa photo à part une jolie frange qui lui
manquait les yeux de peu, des hautes pommettes, des yeux marrons. Elle avait
seulement quatre photo sur son profil, l'une gros plan sur sa figure, les autres
plutôt floues avec des amis en train de faire la fête dont une avec un bouteille
de champagne à la main et un rire moqueur\\

La rencontre virtuelle étant faites, Cédric hésitait à aller faire le premier
pas concret, palpable, celui d'une bière dans un bar; il n'avait pas envie de faire mal
à Jolie. Il n'échangea au début que quelques platitudes en fin de soirées avec
Jade, lorsqu'il prenait une pauses pour se rouler un autre joint dans la nuit
d'hiver. Et puis plus rien, pendant un temps\\


Il fallu attendre que Jolie propose un break en Juillet à leur relation, elle
pensait bientôt retourner en voyage, elle avait l'âme d'une nomade ce quelle ne
disait pas franchement mais qui se devinait, d'abord, l'été dernier avait été le
premier en cinq ans passé à Montréal. Elle voulait ``se trouver'', explorer le monde
et sentait son attachement pour Cédric comme une ancre qui l'empechait de dériver
au gré des courants et vents de la nature. Jolie avait toujours voulu vivre une vie
de navigatrice depuis sa jeunesse a Rimouski où elle voyait les voiles des bateaux
déambuler paresseusement dans le  golf.\\

Cette rupture fut l'effet d'une onde de choc à Cédric. Il ne lui restait pour
l'instant de Jolie que quelques dizaines de pillules d'ecstasy qu'il avait
acheter pour cette dernière lorsqu'elle lui avait demander de passer la commande pour
elle. Il trouvait sa consommation quelque peu excessive mais qui était-il pour
parler de la sorte, lui avec ses joints en forme de canne à pêche.\\

C'est ainsi que Cédric fu surpri un beau soir de fin d'été par Jade qui accepta
son invitation de venir chez lui, à quatre heure du matin, pour se remettre de
la fermeture des bars pour elle, pour s'arracher à ses maths pour lui.\\

Elle avait 22 ans qu'elle disait, Cédric la croyait mais à peine, au creux de
ses aisselles il semblait trouver une jeunesse plus accentuée, une certaine
adolescence qui transpirait de ses pores et dans sa voix, très mince maiss une
rondeur dans l'ossature qui invitait le calinage. Chacun des recoins de son
corps laissaient place à un galbe dans lequel on avait envie de s'éteindre
lentement; poitrine et fesses amples, petit rictus. Elle parlait rapidement,
par petites bourasques. Ce qui la faisait penser inconséquentes.\\

--- scuse moi je dis de la marde \\
--- non je suis pas d'accord, tu te perds tu vas trop vite mais tu dis pas de la marde\\
--- merci tu comprends oui desfois mes mots me perdent en chemin\\

Cédric lui proposa de faire de la MDMA ce qu'elle accepta vivement. Ils
continuèrent à discuter un peu quelque temps histoire de meubler leur soirée
mais vite ils finirent dans les draps à faire l'amour, elle riait parfois dans
les échafourrées, les tortillements, les grincements et les envolées qui
essayaient de suivre le rhythme. Lorsqu'il finirent par atteindre tous les deux
l'orgasme qu'on avait poursuivi sans vouloir le rejoindre trop vite ils
s'allumerent tous les deux une cigarette et jasère de la vie. Elle lui fi part
du fait qu'elle soutenait son mode de vie à toute allure en étant escorte, lui
montra même le site de son agence.


\textit{Cédric développe méta Jade, un programme informatique qui simule leurs interactions
pour peut-être un jour les prédire}


\clearpage

\subsection{cicatrices}

Cédric glisse son doigt sur une cicatrice qu'il a au bras droit. Il ne sait
comment ou pourquoi mais c'est après cette blessure qu'il avait compri, la
résurection, l'impossibilité de se laver l'âme sans mourrir.  Chaque époque lieu
geste laisse une emprunte, aussi infime soit-elle une fois le temps ayant fait
le ménage, il ne le fait jamais complètement on retrouve toujours toutes sortes
de choses sous le tapis. La jalousie en est un bel exemple, cependant bien
d'autres aussi. Il aime Jolie et pourtant il a découvert toutes les coutures de
celui de Jade, c'est une pomme bien croquée, un destin sans nouveau chemin.  Il
faudra l'annoncer à Jolie si jamais ils se revoient, et malgré le fait que ce
fut elle qui demanda un break à leur relation il ne peut s'empêcher de penser,
elle avait des attentes, un modus operandi était annoncé à leur demie-rupture.
Il pourrait faire ce qu'il voulait mais celà n'empecherait pas de faire une
césure à leur relation. Il caresse maintenant une cicatrice que Jade a au dos.
Elle ne veut pas détailler comment cette dernière est arrivée mais il devine le
jeu pervers d'un client. Il a essayé de lui échapper la vérité, de narrativiser
leurs conversations pour en retirer un filet d'explication. Jade connait trop
bien ces petits jeux, elle ne se fera pas avoir. Les cicatrices font mal
malgré le temps, elles nous rappellent l'absence de pardon; l'exorcisme en étant
une bien mauvaise approximation.




\clearpage

\subsection{avouer}

\textit{Au retour de Jolie};\\

Le sort des partenaires de cyprine était difficile pour Cédric à naviguer, ainsi
elle acquieça lorsqu'il lui proposa une sortie à un concert de musique
électronique quelques semaines après son retour du mexique. Elle avait fini par
avoir besoin de changer de pays cet été, que ce ne soit que pour quelques
semaines au plus n'importait pas, l'aventure pouvait se condenser. Elle avait
suivi un rite initiatique pseudo-aztèque où un shaman vous faisait prendre le
mescaline autour d'un feu de camp dans le désert. Avec son amie Florence elle du
se mettre à l'abris des persistenctes sexuelles du shaman mais sommes toute elle
gardait un souvenir positifi de l'expérience.\\

Ils finirent donc deux semaines après l'expérience érotique avec Jade par aller
au concert qui se tenait dans le cadre d'un festival international à danser
ensemble. Cédric, honnête malgré lui disons fini par avouer avoir passé une nuit
avec Jade. Il avait choisi son moment. . Ils avaient tous les deux consommé un
peu d'ecstasy pour aller au concert de musique électronique ou s'enchainaient
entre autre Nicolas Cruz, des percusions foisonnantes de basses bien syncopés
avec des rap plutôt blasés ou illuminés dépendemment du contexte, des riff de
guitare acoustique qui flottent sur les lignes de basses. C'était à la salle de
concert le Métropolis aux coins de St-Laurent et Ste catherine de Montréal, en
bordure du quartier des spectacles. C'est ici que beaucoup d'évennements
d'envergure défilaient mais pour Cédric le lieu lui rappelait tout de même un
cloaque de Montréal. Les itinérants, les illuminés apprenti prophètes s'y
mélaient aux jeune trentenaires bien habillés de t shirt et jeans pour faire la
fête avec l'occasionnelle robe d'été alors que les filles plus novices se
laissaient tanguer sur leur soulier à talon hauts et essayaient de trouver une
démarche qui fasse honneur à leur petites robes serrées.\\

Jolie et Cédric étaient sorti de la salle de danse du Métropolis pour s'accorder
une petite pause de sueur et de musique, ce dernier en profitant aussi pour
s'allumer une cigarette. Il y avait une foule de quelques dizaines de personnes
devant l'entrée, d'autre fêtards faisant la pause sur le troittoir et débordant
dans la rue. Encore bien emmitoufflés dans leur euphorie artificielle les yeux
de Jolie avaient tendance à se détourner vers le vide comme mielleux et rêveurs.\\

-- il faut que je te dise un truc Jolie\\
-- ouais (les yeux encore distraits mais qui se ressaisissent)\\
-- j'ai couché avec Jade la semaine passée, je pensais que tu m'avais laissé pour de bon\\

Jolie avait commencé à afficher le sourire le plus triste qu'il ait jamais vu.

\clearpage


\section{carnets}
----------------------
\begin{center}
--donc plus souple\\
l'air\\
de ses yeux à elle qui sont\\
chez eux \& se dissipent dans\\
un automne de capuches\\
les marées sortent emmitoufflé de paix\\
et/parce que quelqun est la prêt, exprès\\
au complet, peutêtre \\
presque au moins c'est en coin\\
détendu dans une ailleur proche\\
\end{center}

les hublots qui donnent sur le monde\\
il se place sur une plage\\
tiède froide humide salée qui l'ennuie\\
des cils qui lissent le paysage\\
des récifs qui sont beaux\\
pour rien mais avec gloire\\
des goélands caves\\
de la beauté donnée à voir\\
juste assez de monde\\
c'est à dire tout le monde\\
mais différents,bien éparprillés\\
\clearpage
Jolie est partie \\
sans faire un bruit\\
Cédric s'est réveillé\\
sur l'autre esti'oreillé\\
On lui a dit de pas s'en faire\\
que quand même s'tait pas un calvère\\
\\
Le soir Pelleter du bois\\
Après avoir Usé des feux\\
Écrire une chanson pour deux\\
T'expliquer y t'aime pourquoi\\
Se mariner en Acadie \\
se baigner dans une baie\\
s'acheter une perceuse à rabais\\
gosse une adirondak le mardi\\
Chanter une chanson pour deux\\
\clearpage
--------------------
\begin{paracol}{2}
la vie
Dans un cadre de porte\\
m'ennuie\\
affaissé de moitié, fatigué\\
de rien il attent une aube\\
quelque chose qui brille un peu, mais mat quand même\\
du bleu délavé vieux jeans\\
de l'eau de lac qui décape\\
un retour au passé qu'on s'inventerait\\
si ... \\
un ailleurs de chez soi \\
qui cohère, \textit{consistent} et bien pensé\\
\\
\switchcolumn
Cette année ou une autre\\
avant que ça se disloque\\
dans'--  bric a brac du froid écorné \\
on sait pu trop comment ou pourquoi\\
parfois Cédric se force mais\\
le hifi de néon, dla cathodes des arcs
qui shine les spasmes de joies\\
un peu forcées, les colliers fleurit\\
-- trajectoire, y s' ballade dans des réflections de glitter\\
les sons les cris les jouis le pulse
des marées urbaines ou de criquets\\
dans les bar ou les bibliothèques\\
les échanges les pleurs, les crises
le laissent comme une mouette\\
des frites des frites des frites\\
des frites pis rien d'autre\\
caliss y'en revient y y retourne \\
toute scintille, \\
caliss, ça descend mais
de temps en temps ça perce\\
s'en transpirer l'oreillé s'assoupli\\

\end{paracol}
\clearpage
-------------\\
L'air , après un été emmerdant de canicule poisseuse,\\
une brise dans laquelle je berce un utopisme\\
bucolique mais tout de même mouvementé.\\
Le réconfort d'une amour comptatible,\\
en soi cohérent avec nos prédispositions respectives\\
génétiques  ou environnementales\\
qui viennent soit d'une horizon\\
qui me suivrait depuis naissance comme un coucher de soleil d'Escher,\\
ces prédispositions me font rêver pourtant \\
c'est maintenant assez évident\\
plus simple , le moins décadent fanstasme semble effroyament hors de portée.\\
Je connais les étapes les récits les recettes les précipices à enjamber, quelques\\
gens à cotoyer-- des liens à cultiver-- \\
pour parvenir à un certain échafaud
progressivement placé sociétalement, \\
un piedstale contre l'effroi,\\
il ne me manque on dirait qu'un simple assaisonnement bien équilibré\\
de vivacité et de conviction.\\

\clearpage

\section{simulation}

\clearpage

%hiver 1
\section{psychose}
\subsection{changer}
Cédric est en train de finir sa soirée d'étude, donc 4am à tout casser. Il est
dans son bureau, c'est à dire la partie de sa chambre qui surplombe
christophe-colomb, avec vue sur lampadaire jaune par la fenêtre car le bureau
précède la partie chambre, cette dernière en retrait, comme pour être plus
chaleureuse. Deux gros moniteurs sur le bureau, celui-ci en V mais
perpendiculaire à la fenêtre à cadres d'aluminium. Une arche avec moulures
marque la distinction entre la chambre et le reste. \\

message texte de jade:\\
--- allo toujours reveillé :) ?\\
--- oui tu veux venir qu'il répond, maintenant complètement célibataire\\

La chambre de Cédric est un ancien double salon, en avant du côté qui donne sur
la rue, son bureau; direction nord, la porte au dos de la chaise. Jade entre,
d'abord l'appartement puis la pièce double de Cédric. Il l'a senti venir et est
déjà en train de se retourner. Pas besoin de la décrire c'est simplement Vénus
aux cheveux bruns. On a envie de crier pour la dénoncer, t'a pas le droit d'être
belle-dememe on a envie de dire.


--- Cétait bien ta soirée\\
--- bof comme jai closé avec Jessie et ensuite on est allé chez Joe\\
--- hmm\\
--- pis ensuite il nous en restait plus, de la poudre fak le contact à jim s'est pointé\\
--- \ldots\\
--- scuse moi je temmerde avec mes histoires\\
--- non non c'est jusqu'il approche 5am, c'est connu 5 heures c'est les baillements\\
--- ah ouais\\
--- oui c'est convenu, biologique même\\
--- ah ben pas moi\\
--- tu veux quelque chose à boire, j'ai un fond de bière\\
--- oui stp, t a des topes?\\

Cédric lui tend le paquet de 20 mcdonald, king size, cependant il sait qu'il y a
autant de tabac que dans des régulières. La forme importe quand même, les king,
plus longues laissent plus de puff aux obssesifs. Il marche vers le coté ruelle et elle
le suit jusque dans la cuisine. Elle est faite en coin, la table longe le
mur de la porte vers le balcon; rangement à cadavres (de bouteille bien
entendu).

Ils prennent place à la table de la cuisine, collée à la fenêtre
que l'on ouvre légèrement pour s'y placer la gueule avec une cigarette.

--- desfois jade j'ai peur d'être complétement fou\\
--- ben non pour moi t'es la personne la plus sensée que je connais\\
--- j'aimerais ca que t'arrive plus tot desfois\\
--- \ldots\\
--- scuse moi, est ce que tu veux du thé?\\

Et donc cédric qui se lève et active la theiere avec un clic qui disparait
dans la nuit, l'air entre poreusement par la fenêtre, comme un échange
avec la fumée qui ne sait trop où aller Maintenant les deux en fins de soirées
respectives exaspérés par la vie. Jade travaille pour une agence d'escortes
réputés, cosmopolitan de son nom de site web, elle se voit dans l'écran de
cédric lorsqu'elle lui raconte l'histoire de son embauche.\\

Cédric allongé sur le ventre, Jade sur le dos, cote a cote le silence valse,
bientot le telephone cellulaire de Jade sonnera; elle le prendra vivement,
déclarant que c'est son ange gardien qui l'appelle toujours à 5h30, une fois que
son shift fini


--- je sais que je te ferai pas arreter, ta job je veux dire\\
--- merci\\
--- mais si mettons, je sais pas, tu voudrais pas etre serveuse a la place\\
--- j'aime bien m'occuper des gens ta raison\\
--- et je sais, que c'est pas juste, unidirectionnel, je sais pas comment dire\\
mais, c'est plus toi qui me sauve ces temps ci\\
--- faut que t'arrête de penser la\\
--- mais j'aimerais tellement ça tsé,\\
--- ouais je sais\\
--- te sauver\\
--- tu sais j'hais pas ma job tant que ca
--- est ce que ta un bon driver au moins
--- ah ouais ye super vraiment


Un arbre entre le lampadaire et la fenêtre de la chambre, ainsi la lumière
tapisse l'endroit en valsant légèrement. Ils se sont parlés la première fois sur
une application de rencontre l'été passé, on est maintenant en novembre. Dès le
départ Cédric était comme fier de voir Jade au dela de son physique. Elle parle
de façon désordonnée, comme si tout devait jaillir en même temps. Les moins
perspicaces ne voient pas toute la beauté derrière ces mots, comment si ils
prenaient leur temps de se traduire en cohérence il y aurait une poésie qui ne
se traduis pas dans ces mots comme épeurés de sortir trop de vérités\\


\clearpage
\subsection{trauma}

Deux mois plus tard, nous sommes maintenant dans le fond de novembre, les journées disparaissent
avec le soleil qui se couche de plus en plus tôt, ça ne change pas grand chose à la routine
ampthétamine joint de Cédric. D'ailleurs nous sommes la nuit.

Message texte de Jade, 4:20, am bien sur:
--- alloo je sors du Jovers est-ce que je peux venir fumer une clope
--- sure je tattends

Lorsqu'elle arrive elle parle plus brusquement qu'a l'habitude. Elle ploge dans le lit face
première après sa première cigarette dans la chambre de Cédric.
--- le trou de cul le salaud je pensais que je pouvais lui faire à confiance lui\\
--- heyy qu'est ce qui s'est passé.\\
--- je sais pas comment raconter, il insistait mais je voulais pas, il me croyait
pas, ou s'en foutait, il m'a déshabillé pendant que j'essayais de me cacher sous les couvertures\\

Elle parle d'un ton très rapide, ses phrases finissent en écho. Cédric, en peignoir et sweatpants
vient se blottir contre elle. ``tu peux pleurer autant que tu veux je suis là pour ça.''
--- Ça t'ennuit si l'on fait rien ce soir je me sens pas capable\\
--- mais bien sur que c'est correct tu le sais bien allez essaye de dormir\\

Cédric lui caresse tranquillement les cheveux, lui baise le front et retourne
dans la cuisine se faire un autre café, ce soir il ne dort pas, il veille sur
Jade, il a le sentiment que ce qui lui est arrivé est bien plus grand qu'elle ne
laisse savoir. Jade se réveille quelques heures plus tard et ils partagent un autre café.

--- je dois aller me faire tester bientot , c'est un peu une urgence mais j'ai pas envie d'y aller seul\\
--- alors je vais venir moi aussi, je suis du de toute façon, tu sais ou tu veux aller?\\
--- oui, mercredi ça te va.
--- oui tu m'appeleras

Elle quitte


\clearpage
\subsection{coupure}


Mercredi arrive et aucune réponse de Jade aux multiples textos de Cédric. Son
cellulaire vient de répondre mais il ne croit pas que c'est elle. La dernière
fois qu'il a appelé c'est son coloc à elle qui a répondu et il ne lui fait pas
trop confiance. Le personnage a l'air louche, il répond toujours d'un ton
exaspéré. Cédric se rend donc au numéro d'appartement que Jade lui avait remis
lorsquelle lui avait demandé de lui commander un taxi. Cependant c'est un
immeuble à appartement et il ne détient pas le numéro. Il va donc cogner à
chaque porte pour finir au dernier étage. Un type bizarre gros yeux, cheveux par
en arrière collés et poisseux lui répond non ici il y a des Christelles, Sofia
et Jessica mais pas de Jade.\\


Cédric trouve tout celà tout à fait louche, il essaye de rappeler Jade mais elle
ne répond toujours pas. Il s'imagine le pire, encore une fois, et se rappelle
l'air désemparé qu'elle avait fait la dernière nuit qu'elle avait passée chez
lui, à dormir sur le ventre et son maquillage couler tranquillement alors
qu'elle retenait ses pleurs. Cédric fini par appeler la police et leur donner
les informations comme quoi il s'inquiète pour une amie a lui qui ne répond pas
à ces appels, qui s'est fait aggressée récemment et comme quoi le coloc qui a
répondu avait l'air louche. Il les attends une longue heure durant, donne ses
infos et quitte. A l'approche du troittoir d'en face il remarque par des
fenêtres qui donnent sur la cage d'escalier un jeune vêtu d'une tuque qui lui
arrive au ras des oreilles avec des grandes lunettes rondes monté très vite et
redescendre en voyant le char de police Cédric se cache derrière une automobile
en face de l'appartement et prend quelques clichés.\\


\clearpage
\subsection{retailles}

Une autre soirée, même rituel, joint clopes et sex.

Le cellulaire de Jade qui sonne; cette dernière: c'est mon ange gardien!
ce dernier reste anonyme pour Cédric. Elle répond, c'est une voix à accent
Africain qui répond. Selon Jade il est en ce moment en Allemagne; il fait partie
d'une équipe d'arts martiaux qui le fait voyager. Le téléphone est mis en mode
appel conférence

``Je suis chez le gars qui a appelé la cops sur moi''\\
`` ah ouais! vraiment !''\\
``mais Yo ella m'a raconté des histoires qui font peur!''  que doit rétorquer\\
``Cédric, mettre les faits à plat''\\
On entend un rire qui crépite du haut parleur du téléphone de Jade.

\clearpage

\subsection{bresil}

Jade et Cédric marchent sur la rue St-hubert. Arrivés au coin Beaubien, par là
que les commerces commencent à border la rue, ils croisent Fernando et Carlos.
Le premier de Guinée, le deuxième du Brésil, ils demandent quelque information
de touriste; comme où aller prendre un verre. Cédric voit la une occasion de
pratiquer son portugais; langue qu'il avait entâmé d'apprendre après son
décrochage d'études polytechniciennes d'ingénieur, il invitent donc ces deux
nouveaux acolytes à les suivres au Notre dame des quilles, établissement réputé
pour son ouverture d'esprits et ses tendances alternatives.\\

Ils marchent 2 a deux, largeur du trottoir obligeant. Cedric et Fernando
discutent litterature, ce sont deux programmeurs d'ordinateurs dont la veritable
passion est la litterature, c'est une révélation pour eux deux de se retrouve si
proche mentalement ainsi que geographiquement, malgre les continents qui les
séparent.\\

Lorsqu'ils arrivent au bar ils prennent place au comptoir en L. Fernando et
Cedric continuent leur discussion littéraire alors que Carlos et Jade s'effacent
sur le trait inférieur du L. Ces deux premiers partagent la même idée de la vie,
écrire du code informatique parce que ça se vend, alors que la poésie; personne
ne paye pour celà.\\

Cédric surveille du coin de l'oeil Jade et Carlos, il a l'air, sinon de la
dérenger d'être au moins, irritant. Elle se lève après quelque dizaine de minute
pour venir jouer avec Cédric, comment le fait-elle? Et bien elle a l'air d'aimer
lui mettre les mains dans la figure, se retourne danse dos à ventre sur lui,
bref, des simagrées. Cédric continue tant bien que mal sa conversation avec
Fernando\\

En sortant Jade précise à son ami platonique qu'elle n'aime pas le
compère Brésilien, il y a une note dans sa voix qui trahi comme une
espèce de connaissance de l'individu que l'on aurait pas prédit.\\

Le groupe se resepare deux a deux, on se promet de se revoir;
pour ce faire cedric a ajoute fernando comme ami sur facebook.
On voit ainsi qu'il travaille pour la fondation tomas sankara et
pour linux international, un peu de googlage serverait bien; mais
a premiere vue il s'agit la d'un organisme visant a favoriser l'education
sur l'informatique en afrique de l'ouest.\\


\clearpage
















\begin{comment}
%été deux
\section{confinnement}
\subsection{lappart}
cédric a 26 ans et vit tout seul dans un bon gros 4 et demie de biais au marché
atwater par quelques pâtés de maison. Il se molasse de jour en jour à écouter du
jazz et lire les livres de sa bibliothèque du salon qu'il n'avait pas encore lu.
Le bureau était disposé dans sa cuisine avec un ordinateur et deux speakers, un
clavier et une track ball ``comme une souris mais où on fait rouler une grosse
balle sur elle même pour faire bouger le curseur.'' Il fait une diète, histoire de
se débarasser de quelques dizaines de kilos de trop qu'il avait accumulé pendant
l'hiver, un des effets des antidépresseurs qu'il prend chaque matin avec un
espresso très allongé. Il était maintenant passé au Prozac, quelque chose de
plus mainstream que son ancienne médication; il la choisissait lui-même avant,
appelait son psychiatre sa pharmacienne et dans toute chose incluant celle-ci,
il faut dire que Cédric était hipster à cette époque.

celà fait maintenant 6 mois que Jolie l'a quitté, elle avait ses raisons,
des bonnes, un soir elle lui avait dit:\\
--- c'est parce que la vie ne dure pas indéfiniment ce qui confère à chaque moment une valeur infinie\\
--- et moi ceux qui disent ça j'aimerais bien savoir à quel age ils décideraient que ce serait l'heure du suicide, si nous étions immortels
--- mais on est pas immortel\\
--- en tout cas moi si je l'étais j'en serais pas mécontent, existentiellement parlant

\end{comment}












\begin{center}\noindent\rule{0.5\textwidth}{0.4pt}\end{center}
\end{document}
